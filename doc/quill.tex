% !TeX spellcheck = en_US
\documentclass[12pt,a4paper,DIV=calc]{scrartcl}
\usepackage[utf8]{inputenc}
\usepackage[T2A]{fontenc}
\usepackage{amsmath}
\usepackage{amsfonts}
\usepackage{amssymb}
\usepackage{siunitx}
\usepackage{textcomp}
\usepackage[english]{babel}
\usepackage{microtype}
\usepackage{enumitem}
\linespread{1.05}
\KOMAoptions{DIV=last}
\usepackage{mathtools}
\usepackage{physics}
\usepackage{tikz}
\usetikzlibrary{calc}
\usepackage[colorlinks,linkcolor=blue,citecolor=blue,urlcolor=blue]{hyperref}
\frenchspacing

\begin{document}

This document describes the mathematical formulas behind the \textsc{Quill} PIC code and how the formulas used in the lecture notes \cite{PukhovLectures} and the paper \cite{PukhovCERN} by Prof. Pukhov translate to the formulas used here.

\tableofcontents

\newpage

\section{Grid in Quill} \label{sec:grid}

Unlike the grid described in \cite{PukhovLectures, PukhovCERN}, where the charge was in the centers of the cells, in Quill, the charge is in the nodes.

\begin{figure}[h]
\centering
\begin{tikzpicture}
    \def\len{5};
    \def\ldx{2};
    \def\ldy{1};
    \coordinate (A) at (0,0);
    \coordinate (B) at (\len,0);
    \coordinate (C) at ({\len+\ldx},\ldy);
    \coordinate (D) at (\ldx,\ldy);
    \coordinate (A1) at (0,\len);
    \coordinate (B1) at (\len,\len);
    \coordinate (C1) at ({\len+\ldx},{\len+\ldy});
    \coordinate (D1) at (\ldx,{\len+\ldy});
    \draw (A) -- (B);
    \draw (B) -- (C);
    \draw [dashed] (C) -- (D);
    \draw [dashed] (A) -- (D);
    \draw (A) -- (A1);
    \draw (B) -- (B1);
    \draw (C) -- (C1);
    \draw [dashed] (D) -- (D1);
    \draw (A1) -- (B1);
    \draw (B1) -- (C1);
    \draw (C1) -- (D1);
    \draw (A1) -- (D1);
    \node at (A) {\textbullet};
    \node[anchor=north east] at (A) {$\rho^{i,j,k}$};
    \node at (C1) {\textbullet};
    \node[anchor=south west] at (C1) {$\rho^{i+1,j+1,k+1}$};
    \coordinate (AB) at ($(A)!0.5!(B)$);
    \node at (AB) {\textbullet};
    \node[anchor=north] at (AB) {$E_x^{i,j,k}$};
    \coordinate (AC) at ($(A)!0.5!(C)$);
    \node at (AC) {\textbullet};
    \node[anchor=west] at (AC) {$B_z^{i,j,k}$};
    \coordinate (AD) at ($(A)!0.5!(D)$);
    \node at (AD) {\textbullet};
    \node[anchor=south] at (AD) {$E_y^{i,j,k}$};
    \coordinate (AA1) at ($(A)!0.5!(A1)$);
    \node at (AA1) {\textbullet};
    \node[anchor=east] at (AA1) {$E_z^{i,j,k}$};
    \coordinate (AD1) at ($(A)!0.5!(D1)$);
    \node at (AD1) {\textbullet};
    \node[anchor=south] at (AD1) {$B_x^{i,j,k}$};
    \coordinate (AB1) at ($(A)!0.5!(B1)$);
    \node at (AB1) {\textbullet};
    \node[anchor=west] at (AB1) {$B_y^{i,j,k}$};
\end{tikzpicture}
\caption{Depiction of one cell of the grid used in Quill.}
\label{fig:cell}
\end{figure}

\section{Indices used in Quill}\label{sec:indices}
The papers \cite{PukhovLectures, PukhovCERN} use half-integer indices, which is impossible to do in the code.
Below is the table describing how to translate indices from Quill to the indices used in the papers for different fields.

\begin{tabular}{ l | l l | l l }
    Field & Quill &  Papers & Papers & Quill\\
    \hline
    $\rho$ & $i,j,k$ & $\lbrace i,j,k\rbrace-1/2$ & $\lbrace i,j,k\rbrace + 1/2$ & $\lbrace i,j,k\rbrace+1$ \\
    &&&&\\
    $j_x$, $E_x$ & $i,j,k$ & $i,j-1/2,k-1/2$ & $i,j+1/2,k+1/2$ & $i,j+1,k+1$ \\
    $j_y$, $E_y$ & $i,j,k$ & $i-1/2,j,k-1/2$ & $i+1/2,j,k+1/2$ & $i+1,j,k+1$ \\
    $j_z$, $E_z$ & $i,j,k$ & $i-1/2,j-1/2,k$ & $i+1/2,j+1/2,k$ & $i+1,j+1,k$ \\
    &&&&\\
    $B_x$ & $i,j,k$ & $i-1/2,j,k$ & $i+1/2,j,k$ & $i+1,j,k$\\
    $B_y$ & $i,j,k$ & $i,j-1/2,k$ & $i,j+1/2,k$ & $i,j+1,k$\\
    $B_z$ & $i,j,k$ & $i,j,k-1/2$ & $i,j,k+1/2$ & $i,j,k+1$
\end{tabular}

\section{Maxwell solver}

\subsection{The NDFX solver}

\subsubsection{Scheme}

Using the table from Sec.~\ref{sec:indices}, we write the formulas for the NDFX solver with indices used in the PIC code.
As the time scheme is of little interest here, the indices will be in superscripts. 
\begin{align}
    \frac{\Delta B_x^{i,j,k}}{\Delta t} = &-\frac{1}{\Delta y}\big[b_z \left(E_z^{i,j+1,k} - E_z^{i,j,k} \right) + \nonumber \\
    &+a_z \left(E_z^{i,j+1,k+1} - E_z^{i,j,k+1} + E_z^{i,j+1,k-1} - E_z^{i,j,k-1}\right)\big] + \nonumber \\
    &+\frac{1}{\Delta z} \big[b_y \left(E_y^{i,j,k+1} - E_y^{i,j,k}\right) + \nonumber \\
    &+a_y \left(E_y^{i,j+1,k+1}-E_y^{i,j+1,k} + E_y^{i,j-1,k+1} - E_y^{i,j-1,k}\right)
\end{align}
\begin{align}
    \frac{\Delta B_y^{i,j,k}}{\Delta t} = &\frac{1}{\Delta x}\big[b_z \left(E_z^{i+1,j,k} - E_z^{i,j,k} \right) + \nonumber \\
    &+a_z \left(E_z^{i+1,j,k+1} - E_z^{i,j,k+1} + E_z^{i+1,j,k-1} - E_z^{i,j,k-1}\right) - \nonumber \\
    &-\frac{1}{\Delta z} \big[ b_x \left(E_x^{i,j,k+1} - E_x^{i,j,k}\right) + \nonumber \\
    &+a_x \left(E_x^{i+1,j,k+1} - E_x^{i+1,j,k} + E_x^{i-1,j,k+1} - E_x^{i-1,j,k}\right)
\end{align}
\begin{align}
    \frac{\Delta B_z^{i,j,k}}{\Delta t} = &\frac{1}{\Delta y}\big[b_x\left(E_x^{i,j+1,k} - E_x^{i,j,k} \right) + \nonumber \\
    &+a_x \left(E_x^{i+1,j+1,k} - E_x^{i+1,j,k} + E_x^{i-1,j+1,k} - E_x^{i-1,j,k}\right) \big] - \nonumber \\
    &-\frac{1}{\Delta x} \big[ b_y \left(E_y^{i+1,j,k} - E_y^{i,j,k}\right) + \nonumber \\
    &+a_y \left(E_y^{i+1,j+1,k} - E_y^{i,j+1,k} + E_y^{i+1,j-1,k} - E_y^{i,j-1,k} \right) \big]
\end{align}
\begin{align}
    \frac{\Delta E_x^{i,j,k}}{\Delta t} = &\frac{1}{\Delta y}\big[b_z \left(B_z^{i,j,k} - B_z^{i,j-1,k} \right) + \nonumber\\
    &+ a_z \left(B_z^{i,j,k+1} - B_z^{i,j-1,k+1} + B_z^{i,j,k-1} - B_z^{i,j-1,k-1}\right) \big] - \nonumber\\
    &- \frac{1}{\Delta z}\big[ b_y \left(B_y^{i,j,k}-B_y^{i,j,k-1}\right) + \nonumber\\
    &+ a_y \left(B_y^{i,j+1,k} - B_y^{i,j+1,k-1} + B_y^{i,j-1,k} - B_y^{i,j-1,k-1}\right) \big] - \nonumber\\
    &- j_x^{i,j,k}
\end{align}
\begin{align}
    \frac{\Delta E_y^{i,j,k}}{\Delta t} = &-\frac{1}{\Delta x}\big[b_z\left(B_z^{i,j,k} - B_z^{i-1,j,k}\right) + \nonumber \\
    &+ a_z \left(B_z^{i,j,k+1} - B_z^{i-1,j,k} + B_z^{i,j,k-1} - B_z^{i-1,j,k-1}\right)\big] + \nonumber \\
    &+ \frac{1}{\Delta z}\big[ b_x \left(B_x^{i,j,k} - B_x^{i,j,k-1}\right) + \nonumber \\
    &+ a_x\left(B_x^{i+1,j,k} - B_x^{i+1,j,k-1} + B_x^{i-1,j,k} - B_x^{i-1,j,k-1}\right)\big] - \nonumber \\
    &- j_y^{i,j,k}
\end{align}
\begin{align}
    \frac{\Delta E_z^{i,j,k}}{\Delta t} = &\frac{1}{\Delta x}\big[b_y\left(B_y^{i,j,k} - B_y^{i-1,j,k}\right) + \nonumber \\
    &+ a_y \left(B_y^{i,j+1,k} -B_y^{i-1,j+1,k} + B_y^{i,j-1,k} - B_y^{i-1,j-1,k} \right)\big] - \nonumber \\
    &- \frac{1}{\Delta y}\big[ b_x\left(B_x^{i,j,k} - B_x^{i,j-1,k}\right) + \nonumber \\
    &+ a_x \left(B_x^{i+1,j,k} - B_x^{i+1,j-1,k} + B_x^{i-1,j,k} - B_x^{i-1,j-1,k} \right)\big] - \nonumber \\
    &- j_z^{i,j,k}
\end{align}

The coefficients $b_\alpha$ and $a_\alpha$ must satisfy the condition 
\begin{equation}
b_z + 2 a_z = 1
\end{equation}
for the scheme to be valid.

\subsubsection{Dispersion}
\label{sec:maxwell:ndfx:dispersion}

In order to find the dispersion of this scheme, we consider the fields as plane waves,
\begin{align}
    &E_x = E_{x,0} \exp(i \frac{k_x \Delta x}{2}) \exp(-i \omega t + i \vb{k} \vb{r}),\\
    &E_y = E_{y,0} \exp(i \frac{k_y \Delta y}{2}) \exp(-i \omega t + i \vb{k} \vb{r}),\\
    &E_z = E_{z,0} \exp(i \frac{k_z \Delta z}{2}) \exp(-i \omega t + i \vb{k} \vb{r}),\\
    &B_x = B_{x,0} \exp(i \frac{-\omega \Delta t + k_y \Delta y + k_z \Delta z}{2}) \exp(-i \omega t + i \vb{k} \vb{r}),\\
    &B_y = B_{y,0} \exp(i \frac{-\omega \Delta t + k_x \Delta x + k_z \Delta z}{2}) \exp(-i \omega t + i \vb{k} \vb{r}),\\
    &B_z = B_{z,0} \exp(i \frac{-\omega \Delta t + k_x \Delta x + k_y \Delta y}{2}) \exp(-i \omega t + i \vb{k} \vb{r}).
\end{align}
Here, we have taken into account the fact that the field components are displaced both on the grid and in time with respect to each other.

If we use this plane wave, the scheme becomes
\begin{alignat}{2}
    &-A_t B_{x,0} = &- A_y C_z E_{z,0} &+ A_z C_y E_{y,0},\\
    &-A_t B_{y,0} = & A_x C_z E_{z,0} &- A_z C_x E_{x,0},\\
    &-A_t B_{z,0} = & A_y C_x E_{x,0} &- A_x C_y E_{y,0},\\
    &-A_t E_{x,0} = & A_y C_z B_{z,0} &- A_z C_y B_{y,0},\\
    &-A_t E_{y,0} = & -A_x C_z B_{z,0} &+ A_z C_x B_{x,0},\\
    &-A_t E_{z,0} = & A_x C_y B_{y,0} &- A_y C_x B_{x,0},
\end{alignat}
where 
\begin{align}
    &A_t = \frac{1}{\Delta t} \sin(\frac{\omega \Delta t}{2}),\\
    &A_\alpha = \frac{1}{\Delta \alpha} \sin(\frac{k_\alpha \Delta \alpha}{2}),\\
    &C_\alpha = b_\alpha + 2 a_\alpha \cos(k_\alpha \Delta \alpha) = 1 - 4 a_\alpha \Delta \alpha^2 A_\alpha^2.
\end{align}

The equality of the determinant of the system to zero gives us the dispersion relation
\begin{equation}
    A_t^2 = A_x^2 C_y C_z + A_y^2 C_x C_z + A_z^2 C_x C_y.
\end{equation}

\subsection{The FDTD solver}

\subsubsection{Scheme}

The simple FDTD solver is obtained from the NDFX solver by replacing $b_{x,y,z}$ with $1$ and $a_{x,y,z}$ with $0$ and is thus written as

\begin{align}
    &\frac{\Delta B_x^{i,j,k}}{\Delta t} =& -\frac{1}{\Delta y}\left(E_z^{i,j+1,k} - E_z^{i,j,k} \right) +\frac{1}{\Delta z} \left(E_y^{i,j,k+1} - E_y^{i,j,k}\right) \\
    &\frac{\Delta B_y^{i,j,k}}{\Delta t} =& \frac{1}{\Delta x}\left(E_z^{i+1,j,k} - E_z^{i,j,k} \right) - \frac{1}{\Delta z} \left(E_x^{i,j,k+1} - E_x^{i,j,k}\right) \\
    &\frac{\Delta B_z^{i,j,k}}{\Delta t} =& \frac{1}{\Delta y}\left(E_x^{i,j+1,k} - E_x^{i,j,k} \right) - \frac{1}{\Delta x} \left(E_y^{i+1,j,k} - E_y^{i,j,k}\right) \\
    &\frac{\Delta E_x^{i,j,k}}{\Delta t} = &\frac{1}{\Delta y}\left(B_z^{i,j,k} - B_z^{i,j-1,k} \right) - \frac{1}{\Delta z}\left(B_y^{i,j,k}-B_y^{i,j,k-1}\right) &- j_x^{i,j,k}\\
    &\frac{\Delta E_y^{i,j,k}}{\Delta t} =& -\frac{1}{\Delta x}\left(B_z^{i,j,k} - B_z^{i-1,j,k}\right) + \frac{1}{\Delta z}\left(B_x^{i,j,k} - B_x^{i,j,k-1}\right) &- j_y^{i,j,k}\\
    &\frac{\Delta E_z^{i,j,k}}{\Delta t} =& \frac{1}{\Delta x}\left(B_y^{i,j,k} - B_y^{i-1,j,k}\right) - \frac{1}{\Delta y}\left(B_x^{i,j,k} - B_x^{i,j-1,k}\right) &- j_z^{i,j,k}
\end{align}

\subsubsection{Dispersion}
\label{sec:maxwell:fdtd:dispersion}

By applying the dispersion relation in Sec.~\ref{sec:maxwell:ndfx:dispersion} to the FDTD scheme, we get
\begin{multline}
    \frac{1}{\Delta t^2}\sin^2\left(\frac{\omega \Delta t}{2}\right) \\ 
    = \frac{1}{\Delta x^2}\sin^2\left(\frac{k_x \Delta x}{2}\right) + \frac{1}{\Delta y^2}\sin^2\left(\frac{k_y \Delta y}{2}\right) + \frac{1}{\Delta z^2}\sin^2\left(\frac{k_z \Delta z}{2}\right).
\end{multline}
The FDTD scheme is stable if
\begin{equation}
    \frac{1}{\Delta t^2} \leq \frac{1}{\Delta x^2} + \frac{1}{\Delta y^2} + \frac{1}{\Delta z^2}.
\end{equation}

\section{Interpolation of magnetic field on E-field positions}

The magnetic field needs to be interpolated to the positions of the electric field to be used in the particle pusher.
The simplest interpolation (averaging over the 8 closest point) leads to
\begin{multline}
    \hat{B}_x^{i,j,k} = \frac{1}{8} \big(B_x^{i,j,k} + B_x^{i+1,j,k} + B_x^{i,j-1,k} + B_x^{i,j,k-1} + \\
    + B_x^{i+1,j-1,k} + B_x^{i+1,j,k-1} + B_x^{i,j-1,k-1} + B_x^{i+1,j-1,k-1} \big),
\end{multline}
\begin{multline}
    \hat{B}_y^{i,j,k} = \frac{1}{8} \big(B_y^{i,j,k} + B_y^{i-1,j,k} + B_y^{i,j+1,k} + B_y^{i,j,k-1} + \\
    + B_y^{i-1,j+1,k} + B_y^{i-1,j,k-1} + B_y^{i,j+1,k-1} + B_y^{i-1,j+1,k-1} \big),
\end{multline}
\begin{multline}
    \hat{B}_z^{i,j,k} = \frac{1}{8} \big(B_z^{i,j,k} + B_z^{i-1,j,k} + B_z^{i,j-1,k} + B_z^{i,j,k+1} + \\
    + B_z^{i-1,j-1,k} + B_z^{i-1,j,k+1} + B_z^{i,j-1,k+1} + B_z^{i-1,j-1,k+1}\big).
\end{multline}

\section{Shape function}
In PIC methods, the distribution function $f(\vb{r},\vb{p},t)$ is represented as a set of macroparticles
\begin{equation}
    f(\vb{r},\vb{p},t) = \sum_n W_n S(\vb{r} - \vb{r}_n) \delta(\vb{p} - \vb{p}_n).
\end{equation}
In Quill,
\begin{align}
    &S(\vb{r}) = S_x(x) S_y(y) S_z(z),\\
    &S_\alpha(\alpha) = \begin{cases}
        1,&\abs{\alpha} < \Delta \alpha/2,\\
        0,&\abs{\alpha} > \Delta \alpha/2.
    \end{cases}
\end{align}
Here, $\alpha$ can be $x,y,z$.

\section{Density deposition}

The density is calculated as
\begin{equation}
    \rho(\vb{r},t) = \int f(t, \vb{r}, \vb{p}) \dd{\vb{p}} = \sum_n W_n S(\vb{r} - \vb{r}_n).
\end{equation}
The interpolation of this density on the grid is made using the volume integration
\begin{multline}
    \rho^{i,j,k} = \frac{1}{\Delta x \Delta y \Delta z} \int_{x^i-\Delta x/2}^{x^i+\Delta x/2} \int_{y^j-\Delta y/2}^{y^j+\Delta y/2} \int_{z^k-\Delta z/2}^{z^k+\Delta z/2} \rho(\vb{r},t)\dd{t} = \\ 
    =\sum_n W_n S^{\rho}(\vb{r}^{i,j,k} - \vb{r}_n),
\end{multline}
where $\vb{r}^{i,j,k} = (x^i, y^j, z^k) = (i \Delta x, j \Delta y, k \Delta z)$,
\begin{align}
    &S^\rho(\vb{r}) = S_x^\rho(x) S_y^\rho(y) S_z^\rho(z),\\
    &S_\alpha^\rho(\alpha) = \frac{1}{\Delta \alpha} \int_{\alpha-\Delta \alpha/2}^{\alpha + \Delta \alpha/2} S_0(\alpha')\dd{\alpha'} = \begin{dcases}
        1 - \frac{\abs{\alpha}}{\Delta \alpha},& \abs{\alpha} < \Delta\alpha,\\
        0,& \abs{\alpha} > \Delta \alpha.
    \end{dcases}
\end{align}

Hence, a single particle located in the cell shown in Fig.~\ref{fig:cell} induces densities at 8 points
\begin{equation}
\begin{aligned}
    &\rho^{i,j,k} = W b_x b_y b_z, && \rho^{i,j,k+1} = W b_x b_y a_z,\\
    &\rho^{i,j+1,k} = W b_x a_y b_z, && \rho^{i,j+1,k+1} = W b_x a_y a_z,\\
    &\rho^{i+1,j,k} = W a_x b_y b_z, && \rho^{i+1,j,k+1} = W a_x b_y a_z,\\
    &\rho^{i+1,j+1,k} = W a_x a_y b_z, && \rho^{i+1,j+1,k+1} = W a_x a_y a_z,
\end{aligned}
\end{equation}
where $a_\alpha = (\alpha - \alpha^{i,j,k}) / \Delta \alpha$, $b_\alpha = 1 - a_\alpha$.
E.g., if the particle is located at $\vb{r}^{i,j,k}$, then all $a_\alpha = 1$, and the only non-zero density is $\rho^{i,j,k}$.

\section{Currents deposition}

Currents are calculated so that the charge is preserved.
For each current position on the grid, a plane perpendicular to the current direction is taken, and a the charge flow through a square with sides corresponding to the cell size is calculated.
For example, for $j_x$ we have
\begin{equation}
    j_x^{i,j,k} \delta t \Delta y \Delta z = \int_t^{t+\delta t} \dd{t'} \int_{y^j-\Delta y/2}^{y^j + \Delta y /2}\int_{z^k-\Delta z/2}^{z^k + \Delta z /2} \dd{y}\dd{z} v_x \sum_n W_n S(\vb{r}-\vb{r}_n),
\end{equation}
where $\vb{r} = (x^i+\Delta x /2, y, z)$, $\vb{r}_n = \vb{r}_n(t')$.
Now we can calculate the integrals in $y$ and $z$, getting
\begin{equation}
    j_x^{i,j,k} \delta t = \sum_n W_n \int_t^{t+\delta t} \dd{t'} v_x S_x(x^i + \Delta x/2 - x_n) S^\rho_y(y^j - y_n) S^\rho_z(z^k - z_n).
\end{equation}
Similar expressions are obtained for $j_y$ and $j_z$.

Up to that point, the trajectory $\vb{r}_n(t)$ does not matter.
For simplicity, we will consider a single macroparticle which does not leave the $(i,j,k)$ cell.
In a typical scheme, the motion of a particle on a timestep is linear, so that $\vb{r}(t') = \vb{r}_1 + \vb{v} (t'-t)$.
In this case, particle induces currents only in 12 points.
Let us rigorously calculate $j_x^{i,j,k}$, induced by such a particle.
First, 
\begin{equation}
    S_x(x^i + \Delta x/2 - x(t')) = 1,
\end{equation}
because the particle never leaves the cell.
Also, $y^j - y(t') < 0$, $z^k - z(t') < 0$, so the answer is
\begin{multline}
    j_x^{i,j,k} \delta t = W \int_0^{\delta t} \dd{t'} v_x \left(1 - \frac{y_0+v_y t' - y^j}{\Delta y}\right)\left(1 - \frac{z_0+v_z t' - z^k}{\Delta z}\right) = \\
    = W \int_0^{\delta t}\dd{t'} v_x \left(a_y + \frac{v_y (t'-\delta t/2)}{\Delta y}\right)\left(a_z + \frac{v_z (t'-\delta t/2)}{\Delta z}\right) = \\
    = W \delta x \left( b_y b_z + \frac{\delta y\delta z}{12\Delta y \Delta z}\right),
\end{multline}
where 
\begin{equation}
    b_y = 1 - \frac{y_0 + \delta y / 2 - y^j}{\Delta y},\quad b_z = 1 - \frac{y_0 + \delta z / 2 - z^k}{\Delta z}.
\end{equation}

Now we can write down all 12 currents generated by a macroparticle
\begin{equation}
\begin{aligned}
    j_x^{i,j,k} \delta t &= W \delta x (b_y b_z + A_{yz}), & j_x^{i,j+1,k} &= W \delta x (a_y b_z - A_{yz}),\\
    j_x^{i,j,k+1} \delta t &= W \delta x(b_y a_z - A_{yz}), & j_x^{i,j+1,k+1} &= W \delta x(a_y a_z + A_{yz}),\\
    j_y^{i,j,k} \delta t &= W \delta y (b_x b_z + A_{xz}), & j_y^{i+1,j,k} &= W \delta y (a_x b_z - A_{xz}),\\
    j_y^{i,j,k+1} \delta t &= W \delta y (b_x a_z - A_{xz}), & j_y^{i+1,j,k+1} &= W \delta y (a_x a_z + A_{xz}),\\
    j_z^{i,j,k} \delta t &= W \delta z (b_x b_y + A_{xy}), & j_z^{i+1,j,k} &= W \delta z (a_x b_y - A_{xy}),\\
    j_z^{i,j+1,k} \delta t &= W \delta z(b_x a_y - A_{xy}), & j_z^{i+1,j+1,k} &= W \delta z(a_x a_y + A_{xy}),
\end{aligned}
\end{equation}
where
\begin{equation}
    a_\alpha = \frac{\alpha - \alpha^{i,j,k} + 0.5 \delta \alpha}{\Delta \alpha},
    \quad b_\alpha = 1 - a_\alpha,
    \quad A_{\alpha \beta} = \frac{\delta\alpha \delta\beta}{12\Delta\alpha\Delta\beta}.
\end{equation}
If a particle leaves the cell, its trajectory should be separated into several pieces.

\section{Interpolation of fields on particles}
Fields need to be interpolated from the field on particles.
In Quill, the formulas should be
\begin{align}
    &E_x^p = E_x^{i,j+1,k+1} a_y a_z + E_x^{i,j,k+1} b_y a_z + E_x^{i,j+1,k} a_y b_z + E_x^{i,j,k} b_y b_z,\\
    &E_y^p = E_y^{i+1,j,k+1} a_x a_z + E_y^{i,j,k+1} b_x a_z + E_y^{i+1,j,k} a_x b_z + E_y^{i,j,k} b_x b_z,\\
    &E_z^p = E_z^{i+1,j+1,k} a_y a_x + E_z^{i+1,j,k} b_y a_x + E_z^{i,j+1,k} a_y b_x + E_z^{i,j,k} b_y b_x.
\end{align}
Here, $b_{x,y,z} = 1 - a_{x,y,z}$.

\begin{thebibliography}{9}
    \bibitem{PukhovLectures}
    A. Pukhov, Three-Dimensional Particle-in-Cell Simulations of Relativistic Laser-Plasma Interactions. Lecture Notes. (1999).
    
    \bibitem{PukhovCERN}
    A. Pukhov, Particle-In-Cell Codes for Plasma-based Particle Acceleration. \href{https://e-publishing.cern.ch/index.php/CYR/article/view/220}{\textit{CERN Yellow Rep.} \textbf{1}, 181 (2016)}.
\end{thebibliography}

\end{document}